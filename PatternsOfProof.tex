\documentclass{article}

\usepackage{amssymb} %Provides fonts for set symbols & empty set symbol
\usepackage{amsmath}
\DeclareMathOperator{\Exists}{\exists}
\DeclareMathOperator{\Forall}{\forall}
\begin{document}

	\section{Putnam}
	\begin{enumerate}
		\item  Divide and conquer to understand better, i.e. to check if something is divisible, break it up, $\frac{10 + 5}{2} = \frac{10}{2} \frac{5}{2}$, now check, if one is fraction and other is not, then sum is integer + fraction.
		\item A rational $+$ an irrational is irrational.
		\item A rational $\times$ irrational is irrational.
		\item Sometimes breaking up something is not the way to go, instead write it in simplete form, example, $\sqrt{2} + \sqrt{3} + \sqrt{5} = r$.       
		\item Sum of two rationals is always rational.
		\item Apparently either or in math has a beyond terrible definition.
		\item Try to reduce the complexity of the problem by starting from the middle, instead of n, n + 1 , n + 2,  try, n - 1, n, n + 1.
		\item The distance from the origin is a very important function.
	\end{enumerate}
	\section{Predicates}
	\begin{enumerate}
		\item  A predicate can either be quantified to always be true or  sometimes true, the former is known as universally quantified, while the latter as existentially quantified.
			
		\item NOT $(\Exists x, P(x))$ IFF $\Forall x.$ NOT $(P(x))$.
	\end{enumerate}	
	\section{Patterns Of Proofs}
	\subsection{Proof by Contradiction}
\begin{enumerate}

\item A proof by contracdiction is essentially proving the contrapositive of T $\implies$ P, which is, $\neg P \implies F$, this means if we can prove that $\neg P \implies F$, then P must be true.
\item We have to assume the initial statement is false, and take the negation to be true.
\item If a sequance of deduction contradicts the hypothesis then we have an inderect proof.
\item If it contradicts a fact to be known true we have reductio ad absurdum.
\end{enumerate}
\subsection{Proofs about Sets}
\begin{enumerate}
	\item $\in$ means is an element of.
	\item Order does not matter in sets, nor number of times an element appears.
	\item Informally, a set is just a collection of objects, which are called elements.
	\item A set can contain a set.
	\item $\{x, x\} = \{x\}$.
\end{enumerate}

\begin{tabular}{l*{6}{c}r}
Symbol          & Set & Elements\\
\hline
$\varnothing$ & empty set & \\
$\mathbb{N}$ & non-negative integers& \{0, 1, 2, ...\}  \\
$\mathbb{Z}$ & integers& \{..., -1, 0, 1, ....\}  \\
$\mathbb{Q}$ & rational numbers & 0.5, -9, 33.33, ect  \\
$\mathbb{R}$   &  real numbers & $\pi, \sqrt{2}, 9.9$,  ect. \\
$\mathbb{C}$  & complex numbers &  i, 34, ect.  \\
\end{tabular}
\begin{enumerate}
	\item $\mathbb{R}^{+}$ is only positive real numbers.
\end{enumerate}
\end{document}

