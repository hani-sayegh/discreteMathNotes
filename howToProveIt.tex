\documentclass{article}

\usepackage{amssymb} %Provides fonts for set symbols & empty set symbol
\usepackage{amsmath}
\DeclareMathOperator{\Exists}{\exists}
\DeclareMathOperator{\Forall}{\forall}
\begin{document}
	\begin{enumerate}
		\item  $P \lor Q$ is known as a disjunction.
		\item $P \land Q$ is known as a conjuction.
		\item Formulas that are always true are known as tautologies.
		\item Formulas that are always false are known as contradictions.
		\item Equivalent is the same as saying IFF.
	  \item Try to show something is a negative statement with logic.
	  \item An argument is valid if the premisis can not be true with out the conclusion being true.
	  \item Think of a bound variable as a parameter, and a free variable as an argument.
	  \item $\{x | \ldots\}$ binds the variable x.
	  \item$\{x | P(x)\}$ is just the name of a set.
	  \item Keep in mind, $2 \in \{2\}$ but $2 \neq \{2\}$.
	  \item Truth set is the set of all values that makes a statement true.
	  \item Set of all possible values of a variable is known as the universe of discourse for the statement.
	  \item We say variables range over this universe.
	  \item In general, $y \in \{x \in A | P(x)\}$ means the same thing as $y \in A \land P(y)$.
	  \item Tautology means true no matter what.
	  \item Elementhoodtest.
	  \item Null or empty set is represented by $\varnothing$.
	  \item The truth set of a statement P(x) is the set of all values of x that make the statement P(x) true.
	  \item A \textbackslash B is the same as A - B.
	  \item The symmetric differemce: $A \Delta B = (A \setminus B) \cup (B \setminus A) = (A \cup B) \setminus (A \cap B).$
	  \item Two sets are disjoint if they have no elements in common.
	  \item To say $A \land B$ is meaningless.
	  \item Theorem: for any sets A and B, $(A \cup B) \setminus B \subseteq A$.
	  \item The main logical equicalences in logic are:
		  \begin{enumerate}
			  \item DeMorgan's laws.
			  \item Commutative laws.
			  \item Associative laws.
			  \item Idempotent laws.
			  \item Distributive laws.
			  \item Absorption laws.
			  \item Double negation law.
		  \end{enumerate} 
	  \item $y \in \{x | P(x)\} $is the same as $P(y).$
	  \item $\{x \in \mathbb{R} | P(x)\} $ means the universe of discourse can only be the real numbers.
	  \item Truth set of $P(x) = \{x | P(x)\}$, where P(x) is the elementhood test.
	  \item The above is for one free variable only.
	  \item Injective: A one to one function.
	  \item Surjective(onto): A funtion whose image is equal to its codomain.
	  \item Bijective: A function that is both injective and surjective.
	  \item $y \not\in \{x | P(x)\}$ means the same as $\lnot P(y)$.
	  \item Keep in mind $\{x | P(x)\}$ is just the name of a set.
	  \item Logical connectives can only be used to combine statements.
	  \item Set theory operations to combine sets only.
	  \item A is said to be a subset of B if every element in A is in B too, $A \subseteq B$
	\end{enumerate}
	\subsection{The Conditional and Biconditional Connectives}
	\begin{enumerate}
		\item If P then Q is denoted by $P \implies Q$, this known as a conditional statement, where P is the antecedent and Q is the consequent.
		\item $P \implies Q$ and $\lnot P \lor Q$ and $ \lnot (P \land Q)$ are all the same.
            \item $P\Leftrightarrow Q$ means the biconditional.

	\end{enumerate}
    \section{Quantificational Logic}
    \begin{enumerate}
    \item $\forall x P(x) \rightarrow Q(x) $means $(\forall x P(x)) \rightarrow Q(x), $not $\forall x(P(x) \rightarrow Q(x)).$
        \item We say a universal quantifier binds a variable.
            \item A conditional statement means implies.
    \end{enumerate}
\end{document}



